\documentclass{scrartcl}
\usepackage{listings}
\usepackage{xcolor}
\usepackage{amssymb}
\usepackage[colorlinks=true, urlcolor=blue, linkcolor=red]{hyperref}
\usepackage{graphicx}

\definecolor{lightcyan}{HTML}{E0FFFF}




\begin{document}
    \lstset{
        language=Java,
        %numbers=left,
        stepnumber=1,
        numbersep=5pt,
        backgroundcolor=\color{lightcyan},
        showspaces=false,
        showstringspaces=false,
        showtabs=false,
        tabsize=2,
        captionpos=b,
        breaklines=true,
        breakatwhitespace=true,
        title=\lstname,
        basicstyle=\small
    }



\section{Data Types}

\subsection{Date and Time}

\subsubsection{\lstinline$ LocalDate $}

    LocalDate is an immutable date-time object that represents a date, often viewed as year-month-day. Other date fields, such as day-of-year, day-of-week and week-of-year, can also be accessed.

    \begin{lstlisting}
        // to obtain, e.g.
        static LocalDate of(int year, int month, int dayOfMonth)
        static LocalDate of(int year, Month month, int dayOfMonth)
        static LocalDate ofInstant(Instant instant, ZoneId zone)
        static LocalDate parse(CharSequence text, DateTimeFormatter formatter)

        // instance methods, e.g.
        LocalDateTime atTime(int hour, int minute, int second, int nanoOfSecond)
        LocalDateTime atTime(LocalTime time)

        int getDayOfMonth()
        DayOfWeek getDayOfWeek()
        int getDayOfYear()
        Month getMonth()
        int getMonthValue()
        int getYear()

        // same for plus
        LocalDate minus(long amountToSubtract, TemporalUnit unit)
        LocalDate minusDays(long daysToSubtract)
        LocalDate minusMonths(long monthsToSubtract) //etc
    \end{lstlisting}

\subsubsection{\lstinline$ LocalTime $}

    LocalTime is an immutable date-time object that represents a time, often viewed as hour-minute-second. Time is represented to nanosecond precision. For example, the value "13:45.30.123456789" can be stored in a LocalTime.

    \begin{lstlisting}
        // to obtain, e.g.
        static LocalTime of(int hour, int minute, int second, int nanoOfSecond)
        static LocalTime ofInstant(Instant instant, ZoneId zone)

        // instance methods, e.g.
        LocalDateTime atDate(LocalDate date)

        int getHour()
        int getMinute() //etc.

        // same for minus
        LocalTime plus(long amountToAdd, TemporalUnit unit)
        LocalTime plusNanos(long nanosToAdd) // etc.

        // returns copy
        LocalTime withHour(int hour)
        LocalTime withMinute(int minute) //etc.
        \end{lstlisting}

\subsubsection{\lstinline$ LocalDateTime $}

    \begin{lstlisting}
        // to obtain, e.g.
        static LocalDateTime of(int year, Month month, int dayOfMonth, int hour, int minute, int second, int nanoOfSecond)
        static LocalDateTime of(LocalDate date, LocalTime time)
        // instance methods analogous to above
    \end{lstlisting}

\subsubsection{\lstinline$ Month $}

    In addition to the textual enum name, each month-of-year has an int value (1-12).
    Do not use ordinal() to obtain the numeric representation of Month. Use getValue() instead.

    \begin{lstlisting}
        // to obtain, e.g.
        static Month of(int month)
        Month e = Month.of(10); // DECEMBER
        static Month valueOf(String name)
        Month m = Month.valueOf("DECEMBER"); // DECEMBER

        // instance methods, e.g.
        int getValue()
        int length(boolean leapYear)
        minus(long months)
        plus(long months)
    \end{lstlisting}

\subsubsection{\lstinline$ ChronoUnit $}

    \begin{lstlisting}
        // to obtain, e.g.
        static ChronoUnit valueOf(String name)

        // instance methods, e.g.
        <R extends Temporal> R addTo(R temporal, long amount) // returns a copy!
        long between(Temporal temporal1Inclusive, Temporal temporal2Exclusive)
\end{lstlisting}

\subsubsection{\lstinline$ Instant $}

     An \lstinline$ Instant $ represents a specific moment in time using GMT.
    Consequently, there is no time zone information.

    \begin{lstlisting}
        // to obtain, e.g.
        static Instant from(TemporalAccessor temporal)
        static Instant now()
        static Instant ofEpochMilli(long epochMilli)

        // instance methods, e.g.
        OffsetDateTime atOffset(ZoneOffset offset)
        ZonedDateTime atZone(ZoneId zone)

        Instant minus(long amountToSubtract, TemporalUnit unit) //returns copy! others too
        Instant minus(TemporalAmount amountToSubtract)

        Instant minusMillis(long millisToSubtract)
        Instant minusNanos(long nanosToSubtract)

        var instant = trainDay.toInstant(); // will not compile if this is a LocalDateTime!

    \end{lstlisting}

\subsubsection{\lstinline$ Period $}

    This class models a quantity or amount of time in terms of years, months and days. See Duration for the time-based equivalent to this class.

    Durations and periods differ in their treatment of daylight savings time when added to ZonedDateTime. A Duration will add an exact number of seconds, thus a duration of one day is always exactly 24 hours. By contrast, a Period will add a conceptual day, trying to maintain the local time.

    For example, consider adding a period of one day and a duration of one day to 18:00 on the evening before a daylight savings gap. The Period will add the conceptual day and result in a ZonedDateTime at 18:00 the following day. By contrast, the Duration will add exactly 24 hours, resulting in a ZonedDateTime at 19:00 the following day (assuming a one hour DST gap).

    The supported units of a period are YEARS, MONTHS and DAYS. All three fields are always present, but may be set to zero.

    The period is modeled as a directed amount of time, meaning that individual parts of the period may be negative.

    \begin{lstlisting}
        // to obtain, e.g.
        static Period between(LocalDate startDateInclusive, LocalDate endDateExclusive)
        static Period of(int years, int months, int days)
        static Period ofDays(int days) // other fields will be 0
        static Period ofMonths(int months)

        // instance methods, e.g.
        Temporal addTo(Temporal temporal)
        Period minusDays(long daysToSubtract) // all return copies!
        minusMonths(long monthsToSubtract)

        Period withMonths(int months) // copies, too!
        Period withYears(int years)

        int getDays()
        \end{lstlisting}

 \subsubsection{\lstinline$ Duration $}

    This class models a quantity or amount of time in terms of seconds and nanoseconds. It can be accessed using other duration-based units, such as minutes and hours. In addition, the DAYS unit can be used and is treated as exactly equal to 24 hours, thus ignoring daylight savings effects.

    See Period for the date-based equivalent to this class.

     \begin{lstlisting}
     // to obtain, e.g.
     static Duration of(long amount, TemporalUnit unit)
     static Duration ofDays(long days)

     // instance methods, e.g.
     Duration dividedBy(long divisor) // all these copy
     long dividedBy(Duration divisor)

     long get(TemporalUnit unit)
     int getNano()
     long getSeconds()
     \end{lstlisting}

 \subsection{String and StringBuilder}
 \subsubsection{\lstinline$ String $}

    \begin{lstlisting}
     // strip()-related methods (these are the only ones)
     strip(), stripLeading(), stripTrailing(), stripIndent()

     // indent(): normalizes the output by adding a line break to the end
     // does not change the indentation, but still adds a normalizing line break
     System.out.println(phrase.indent(0).length());

     // translateEscapes()
     // these print 2 lines:
     System.out.println("cheetah\ncub");
     System.out.println("cheetah\ncub".translateEscapes());
     System.out.println("cheetah\\ncub".translateEscapes());
     - this prints 1:
     System.out.println("cheetah\\ncub");

     // format string
     var quotes = """
     \"The Quotes that Could\" // could remove both backslashes
     \"\"\"                    // could remove 2 backslashes
     """;

     // there is no reverse()
     \end{lstlisting}

 \subsubsection{\lstinline$ StringBuilder $}

    \begin{lstlisting}

        // instance methods, e.g.
        char charAt(int index)
        IntStream chars()

        int indexOf(String str)

        int length()

        StringBuilder
        delete(int start, int end)
    \end{lstlisting}

\subsection{Numbers}
\subsubsection{\lstinline$ Math $ methods}
    \begin{lstlisting}
        Math.round() // double -> double
        Math.max() // overloaded, returns passed-in type
        Math.pow() // double -> double
    \end{lstlisting}

\subsubsection{Parsing Strings}
    \begin{lstlisting}
        var numPigeons = Long.parseLong("100"); // returns long
        var numPigeons2 = Long.valueOf("100"); // returns Long
    \end{lstlisting}

    Examples:
    \begin{lstlisting}
        Boolean.valueOf("8").booleanValue() // false
        Character.valueOf('x').byteValue(); // does not compile
        Double.valueOf("9_.3").byteValue(); // NumberFormatException
        Long.valueOf(128).byteValue(); // - 128
    \end{lstlisting}

\section{Operators}
\subsection{Kinds}
\subsubsection{Logical Operators}
    \begin{lstlisting}
        &&
        ||
        // no ~!
    \end{lstlisting}

\subsubsection{Bitwise Operators: Logical}
    \begin{lstlisting}
        &
        int result = 5 & 6; // 4    101 & 110 = 100
        |
        int result = 5 | 6; // 7    101 | 110 = 111
        ^
        int result = 5 ^ 6; // 3    101 ^ 110 = 011
        ~
        int result = ~6; // -7      0000 0110 -> 1111 1001 -> 0000 0110 + 1 -> 0000 0111
    \end{lstlisting}

    The bitwise NOT or complement operator is equivalent to negation of each bit in the input value. This will result in a negative number one smaller, i.e., obtain -x-1 from x.

    Steps: First we need to find its 2’s complement, and then convert the resultant binary number into a decimal number.

    \begin{enumerate}
        \item write 6 in binary: 0000 0110
        \item take complement: 1111 1001 // this is the 1's complement
        \item to get the 2's complement (since numbers are stored as 2’s complement), add 1: 1111
    \end{enumerate}

    To find the binary representation of -17, take the 2's complement of 17:
    \begin{enumerate}
        \item 17 = 0001 0001
        \item Take the bitwise complement: 1110 1110
        \item Add 1: 1110 1110 + 1 = 1110 1111
    \end{enumerate}

    To take the 2's complement of negative number:
    \begin{enumerate}
        \item Start from binary -17: 1110 1111
        \item     Take the the bitwise complement: 0001 0000
        \item     Add 1: 0001 0001
        \item     This gets back the 17!
    \end{enumerate}

    To find the decimal representation of a number given in binary, reverse steps
   \begin{enumerate}
       \item  Subtract 1: 1110 1111 - 1 = 1110 1110
       \item     Take the complement of the complement: 0001 0001
       \item     Change from base 2 back to base 10 16 + 1 = 17
       \item     Rewrite this as a negative integer: -17
   \end{enumerate}

\subsubsection{Bitwise operators: arithmetic}
    \begin{lstlisting}

        << // shift left (signed)
        >> shift right (signed)
        >> shift right (unsigned)

        12 << 2: 48 // *2^n
        12 >> 2: 3 // 1100 -> 0011 (pos.: fill with 0)
        -12 >> 2: -3 //               (neg..: fill with 1)
        12 >>> 2: 3 // 1100 -> 0011
        -12 >>> 2: 1073741821 // fill with 0 too
    \end{lstlisting}

\subsection{Precedence}

    Default evaluation order is left-to-right.

    \begin{tabular}{|l|c|r|}
        \hline
        Post-­unary operator& x++, x-- &  \\
        \hline
        Pre-­unary operator& ++x, ++x &  \\
        \hline
        Other unary operators& -­, !, ~, + &  Right-­to-­left \\
        \hline
        Cast(type)reference&  &  Right-­to-­left \\
        \hline
        Multiplication/division/modulus& *, /, \% &  \\
        \hline
        Addition/subtraction& +, - &  \\
        \hline
        Shift operators& $\ll, \gg, \ggg$ &  \\
        \hline
        Relational operators & $<$, $>$, $<=$, $>=$, instanceof & \\
        \hline
        Equal to/not equal to& ==, != &\\
        \hline
        Logical and& \& &  \\
        \hline
        Logical exclusive OR& \textasciicircum \\
        \hline
        Logical inclusive OR& $|$ &  \\
        \hline
        Conditional OR& $||$ &  \\
        \hline
        Ternary operator& e1 ? e2 : e3 & Right-­to-­left\\
        \hline
        Assignment operators& =, +=, -­=, *=, /=, \%=, \&=, \textasciicircum, & Right-­to-­left\\
        & $|$, $<<=$ , $>>=$, $>>>=$ & \\
        \hline
        Ternary operator& e1 ? e2 : e3 & Right-­to-­left\\
        \hline
        Arrow operator& $->$ &  \\\\
        \hline
    \end{tabular}
     \\

    In a nutshell:

    \begin{itemize}
        \item shift ops after +, -
        \item relational before equality before logical
        \item \& $|$ before \&\& $||$
        \item ternary thereafter but before assignment
        \item assignment last
    \end{itemize}

\section{Syntax}
\subsection{switch}

\subsubsection{General}

        \begin{itemize}
            \item Supports: enum, byte, Byte, short, Short, char, Character, int, Integer,
            String, var (if resolves to one of those types)
            \item Does not support: boolean, Boolean, double, Double, float, Float
         \end{itemize}

\subsubsection{Statement}

    \begin{itemize}
        \item The value of a case statement must be a constant, a literal value, or a final variable (not, e.g., \lstinline$ case red:  $)
        \item May use comma to separate case constants in statements: e.g.,
            \begin{lstlisting}
                public int getAverageTemperate(Season s) {
                    switch (s) {
                        default:
                        case WINTER, SUMMER: return 30;
                }}
            \end{lstlisting}
    \end{itemize}

\subsubsection{Expression}

    \begin{itemize}
        \item Can omit default clause in when either all the values of an enum are covered or no value is returned
    \end{itemize}

\subsection{ for (x : y)}

     A for-­each loop accepts arrays and classes that implement java.lang.Iterable, such as List. Not: String, StringBuilder

\subsection{Flow Scoping}

    \begin{lstlisting}
        if (number instanceof Integer i && Math.abs(i) == 0) // ok
        if (number instanceof Integer i || Math.abs(i) == 0) // i not defined
        if (number instanceof int i && Math.abs(i) == 0) // can't use primitives
    \end{lstlisting}

\subsection{Loops}

    for and while (but not do-while) don't need braces if just one statement follows.

    \begin{lstlisting}
        while (i<6) System.out.println("");
        for (;;) System.out.println();
    \end{lstlisting}

\section{Classes, Interfaces, Records and Enums}
\subsection{Classes}
\subsubsection{Nested Classes}

    There are four types of nested classes: inner, static, local, and anonymous.
    Nested classes can be public.

    An \textit{inner} class requires an instance of the outer class to use.
    Three ways that are legal:

    \begin{lstlisting}
    public class Dinosaur {
        class Pterodactyl extends Dinosaur {}

        // it all happens in an instance method
        public void roar() {
            var dino = new Dinosaur();

            // uses instance to create inner
            dino.new Pterodactyl();

            // relies on the fact that roar() is an instance method
            //, which means there's an implicit instance of the outer class Dinosaur available
            new Dinosaur.Pterodactyl();

            // The Dinosaur. prefix is optional, though
            new Pterodactyl();
        } }
    \end{lstlisting}

    While a \textit{static} nested class does not:

    \begin{lstlisting}
        new Lion.Den()
    \end{lstlisting}

    A \textit{local} class is commonly defined within a method or block.
    Local classes can only access local variables that are final or effectively final.

    \textit{Anonymous} classes are a special type of local class that does not have a name.
    Anonymous classes are required to extend exactly one class or implement one interface.

    Note:

    \textit{Inner, local, and anonymous} classes can \textit{access private members} of the class in which they are defined, provided the latter two are used inside an instance method.

    \textit{All four} types of nested classes can now define static variables and methods.

\subsubsection{Sealed Classes}

    A sealed class is a class that restricts which other classes may directly extend it:

    \begin{lstlisting}
         sealed class Friendly extends Mandrill permits Silly {}
    \end{lstlisting}

    Parent and child must be in same package. If they are in same file or the extension is nested, no permits are needed.

    Every class that directly extends a sealed class must specify exactly one of the following three modifiers: final, sealed, or non-­sealed.

    Note:

    - We can have sealed interfaces (permitting both extensions and implementations).
    - While a sealed class is commonly extended by a subclass marked final, it can also be extended by a sealed or non-­sealed subclass marked abstract.

    \begin{lstlisting}
        // extends clause missing; it is Friendly that does not compile!
        sealed class Friendly extends Mandrill permits Silly {}
        final class Silly {}
    \end{lstlisting}
\subsubsection{Final and Immutable Classes}

    - Classes marked as final can’t be extended.
    - Immutable classes do not include setter methods. They must be marked final \textit{or} contain only private constructors.

\subsection{Interfaces}

    \textit{Variables} are always publics, static, final.

    \textit{Methods} can be either \textit{default} (these are always public), static (always public), private, or private and static. There are no protected members.

    \textit{Non-­static} methods with a body must be explicitly marked \textit{private} or \textit{default}.

    Note:

    - There is no modifier that can prevent a default method from being overridden in a class implementing an interface!

    - If implementing two interfaces with conflicting signatures, \textit{default} methods have to override it explicitly. They  can then access the inherited ones by calling Interface.super.method().

    - \textit{Static} methods are only accessible with a qualifier.

\subsection{Records}

    Minimal example:

    \begin{lstlisting}
        public record Crane(int numberEggs, String name) { }
    \end{lstlisting}

    Records may optionally have constructors.

    A (short) constructor (at most one):

    \begin{lstlisting}
        public Crane { // no parens
            if (numberEggs < 0) throw new IllegalArgumentException();
            name = name.toUpperCase();
            // long form is automatically called here
        }
    \end{lstlisting}

    A long constructor:

    \begin{lstlisting}
        public Crane(int numberEggs, String name) {
            if (numberEggs < 0) throw new IllegalArgumentException();
            this.numberEggs = numberEggs;
            this.name = name;
        }
    \end{lstlisting}

    Constructors may be overloaded:

    \begin{lstlisting}
        public record Crane(int numberEggs, String name) {
            public Crane(String firstName, String lastName) {
                // must be 1st call, must either call another constructor or the long constructor
                this(0, firstName + " " + lastName);
            }
        }
    \end{lstlisting}

    They can also implement interfaces:

    \begin{lstlisting}
        public record Crane(int numberEggs, String name) implements Bird {}
    \end{lstlisting}

    Overriding a method:

     \begin{lstlisting}
        public record BeardedDragon(boolean fun) {
            // overriding generated accessor
            @Override public boolean fun() { return false; }
        }
    \end{lstlisting}

    Records may be package access or public. They may contain methods, nested classes, interfaces, annotations, enums, and other records.

    Records can't be subclassed, since they are implicitly final.

    Records \textit{cannot} declare instance \textit{variables} (as opposed to instance \textit{methods}) or instance \textit{initializers}.

\subsection{Enums}

    Enums can have constructors, methods, and fields. Example:

        \begin{lstlisting}
        enum Flavors {
            VANILLA, CHOCOLATE, STRAWBERRY;
            static final Flavors DEFAULT = STRAWBERRY;
        }
    \end{lstlisting}

    Constructors are implicitly private. Example:

     \begin{lstlisting}
        enum Animals {
              // When an enum contains any other members, such as a constructor or variable, a semicolon is required:
            MAMMAL(true), INVERTEBRATE(Boolean.FALSE), BIRD(false),REPTILE(false),
            AMPHIBIAN(false), FISH(false) {public int swim() { return 4; }};

            final boolean hasHair;

            private Animals(boolean hasHair) {this.hasHair = hasHair;}

            public boolean hasHair() { return hasHair; }
            public int swim() { return 0; }
        }

    \end{lstlisting}

\subsection{Overriding and overloading}

\subsubsection{Overloading}

    Overloading works also for pairs of primitive + wrapper, e.g.

    \begin{lstlisting}

        public static void main(String... args) {
            System.out.println(new App().woof(5)); // 1
            System.out.println(new App().woof(Integer.valueOf(5))); // 2
        }
        public String woof(int bark) {
            return "1";
        }
        public String woof(Integer bark) {
            return "2";
        }
    \end{lstlisting}



\subsubsection{Overriding}

    Overridden and hidden methods can only have covariant return types.
    This also applies to implementing abstract methods.

    When a parent method is private, no overriding takes place (so the child can do what it wants).

    Note:

    - Overriding replaces the method \textit{regardless of the reference type}.
    - There is no overriding of instance \textit{variables}! Instance variables are always determined based on the reference type!

\section{IO}
\subsection{java.nio}
\subsection{java.io}

\subsubsection{Constructing a path}

    To construct a path, use Path.of or Paths.get:

    \begin{lstlisting}
        Path zooPath1 = Path.of("/home/tiger/data/stripes.txt");
        Path zooPath2 = Path.of("/home", "tiger", "data", "stripes.txt");

        Path zooPath3 = Paths.get("/home/tiger/data/stripes.txt");
        Path zooPath4 = Paths.get("/home", "tiger", "data", "stripes.txt");
    \end{lstlisting}

\subsubsection{Conversion to or back from java.io.File}

    \begin{lstlisting}
        File file = new File("rabbit");
        Path nowPath = file.toPath();
        File backToFile = nowPath.toFile();
    \end{lstlisting}

\subsubsection{Concatenation: Path resolve(Path other)}

    Resolves the given path against this path. Does not normalize!

    \begin{lstlisting}
        // the input argument is appended onto the Path, e.g.

        // with input a relative path:
        Path path1 = Path.of("/cats/../panther");
        Path path2 = Path.of("food");
        System.out.println(path1.resolve(path2));
        // /cats/../panther/food

        // if input is absolute, return input
        Path path3 = Path.of("/turkey/food");
        path3.resolve("/tiger/cage");
        // /turkey/food
    \end{lstlisting}

\subsubsection{Constructing a relative path: Path relativize(Path other)}

    \begin{lstlisting}
        //requires that both path values be absolute or relative
        // otherwise, an exception is produced at runtime
        var path1 = Path.of("fish.txt");
        var path2 = Path.of("friendly/birds.txt");
        path1.relativize(path2);
        // ../friendly/birds.txt

        // Note: the file itself counts as one level!
        path2.relativize(path1);
        // ../../fish.txt
        // go up "plus 1"
    \end{lstlisting}

\subsubsection{Normalizing a path}

    \begin{lstlisting}
        var p1 = Paths.get("/pony/../weather.txt");
        var p2 = Paths.get("/weather.txt");
        p1.equals(p2); // false
        p1.normalize().equals(p2.normalize()); // true
    \end{lstlisting}

\subsubsection{Resolve symlinks: toRealPath}

    \begin{lstlisting}
        ll horse/
        schedule/
        food.txt

        ll zebra/
        schedule/
        food.txt

        Paths.get("/zebra/food.txt").toRealPath(); // /horse/food.txt
        Paths.get(".././food.txt").toRealPath());  // same -- normalizes
    \end{lstlisting}

\subsubsection{Constructing a path}

    \begin{lstlisting}

    \end{lstlisting}

\subsubsection{Common method arguments}

    \begin{lstlisting}

        // Enums implementing (empty) interfaces, e.g.
        public enum StandardCopyOption implements CopyOption {
            REPLACE_EXISTING,
            COPY_ATTRIBUTES,
            ATOMIC_MOVE;
            private StandardCopyOption() {}}

    \end{lstlisting}








\end{document}

